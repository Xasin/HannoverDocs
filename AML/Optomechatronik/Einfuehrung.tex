
Die Optomechatronik ist ein sehr modernes und weitreichendes Forschungsgebiet. Die Anwendung optomechatronischer Systeme spannt sich von simpler Beleuchtung über Projektion und Informationsübertragung bis hin zu optischen Bearbeitungsverfahren für Mikro- und Nanostrukturen. ``Optomechatronik'' steht hierbei für die Kombination aus Optik, Informatik, Elektronik und Mechanik. Ingenieure welche sich mit diesen Technologien befassen müssen also ein breites Feld an Technologien beherrschen.

Der Versuch ``Kleine Laborarbeit Optomechatronik'' soll den Teilnehmenden einen ersten Einblick in dieses Themengebiet ermöglichen, und ihnen wichtige Grundlagen zur Verwendung und Analyse optomechatronischer Systeme vermitteln. Als Beispiel wird hierfür ein handelsüblicher LCD-Beamer verwendet, dessen einzelne Bestandteile von den Teilnehmenden separat mit optischen Messverfahren untersucht werden. Auch wichtig ist hierbei die Betrachtung der menschlichen Farbwahrnehmung, und wie der Beamer für Zuschauende ein nahezu durchgängiges Farbspektrum generieren kann.

Zusätlich werden alternative Technologien wie z.B. der One-Chip-DLP Beamer angesprochen, und es werden die Messverfahren selbst genauer erklärt, um den Teilnehmenden ein eigenständiges Ausführen, Auswerten und Korrigieren der folgenden Versuche zu ermöglichen.

