
In den ersten zwei Teilversuchen wurden die für die Farbgebung eines Beamers wichtigen Bauteile untersucht. Der dritte Teilversuch befasst sich nun mit der Analyse eines vollständigen Beamers und dessen Farbwiedergabe, um reale Werte für den Vergleich mit den berechneten Erwartungswerten zu liefern.

Mithilfe eines Goniometers und einer Leuchtdichtekamera werden die drei Farbkomponenten des Beamers über dessen Projektionsfläche gemittelt berechnet, um den Teilnehmenden eine Berechnung des realen Farbraumes zu ermöglichen.

\subsubsection{Versuchsaufbau}

Der Aufbau besteht aus einem Goniometer, d.h. einer automatisch drehbaren Platform, auf welche der Beamer befestigt ist. Der Beamer ist auf eine Leuchtdichtekamera gerichtet, welche mithilfe verschiedener Filter die Rot-, Grün- und Blauanteile des vom Beamer ausgestrahlten Lichtes misst. Ein Computer wird verwendet, um die Spektren der Leuchtdichtekamera zu speichern und um den Beamer und das Goniometer anzusteuern.

\subsubsection{Versuchsdurchführung}

Die Messung der drei Farbkomponenten wird in den folgenden Schritten durchgeführt:
\begin{itemize}
\item Die Belichtungszeit der Leuchtdichtekamera wird für die drei Farben so bestimmt, dass nicht im Sättigungsbereich des Sensors gemessen wird. Für die Teilnehmenden wurden bereits bestimmte Werte für die drei Farben vorgegeben.
\item Es werden 15 Rotationen des Goniometer, welche möglichst gleichmäßig über die Projektionsfläche des Beamers verteilt sind, ausgewählt. Die Teilnehmenden wählten Winkel von $+15^\circ$; $\pm2.5^\circ$ aus.
\item Es wird ein PDF einer der drei Farben (Rot, Grün oder Blau) über den Beamer projiziert, und für die gewählte Farbe über die 15 Goniometerrotationen gemittelt der Farbpunkt gemessen. Der berechnete Farbpunkt wird abgespeichert, und dieser Schritt wird für alle drei Farben ausgeführt.
\end{itemize}

Zu beachten ist, dass das PDF mit den Farbbildern nicht unbedingt 100\% die Farbkanäle des Beamers ansteuert. Es gibt so z.B. immer eine gewisse Farbraumumrechnung, welche die Messwerte der Farbpositionen beeinträchtigen kann. Quantifizert werden kann dies unter Kenntnis des für das PDF verwendeten Farbraums, des Farbraumes des Beamers selbst, sowie eventueller Kalibrationsparameter des Beamers, die hier durch fehlende Spezifikation nicht genauer betrachtet werden können. Auch wird die Standardabweichung der einzelnen Farbwerte berechnet, um die Varianz der Farben quantifizieren zu können. Der gemessene durchschnittliche Farbraum, berechnet nach [\cite[Seite 27]{AML_SKRIPT}], ist in Tabelle \ref{tab:TV3_Averages} sowie in der Auswertung in Abschnitt \ref{A_FCOORDS} zu sehen.

\begin{eqnarray*}
x_{n,m} = &\frac{1}{N}\sum_{k=1}^{N} x_{n,k} \\
y_{n,m} = &\frac{1}{N}\sum_{k=1}^{N} y_{n,k} \\
x_{n,\sigma} = &\sqrt{\frac{1}{N-1}\sum_{i=1}^{N}\left(x_{n,i}-x_{n,m}\right)}  \\
y_{n,\sigma} = &\sqrt{\frac{1}{N-1}\sum_{i=1}^{N}\left(y_{n,i}-y_{n,m}\right)}
\end{eqnarray*}

\begin{table}[h!]
\centering
\caption{Statistische Werte der Farbraummessung}
\label{tab:TV3_Averages}
\begin{tabular}{| c | c | c | c | c |}
\hline
n & $x_{n,m}$ & $x_{n,\sigma}$ & $y_{n,m}$ & $y_{n,\sigma}$ \\
\hline
Rot (1) & 0.4263 & 0.02422 & 0.4545 & 0.02037 \\
Grün (2) & 0.3165 & 0.00284 & 0.5949 & 0.00307 \\
Blau (3) & 0.1876 & 0.00659 & 0.2227 & 0.02946 \\
\hline
\end{tabular}
\end{table}

\subsubsection{Auswertung}

In Abbildung \ref{A_Farbdreieck} sind die drei gemessenen gemittelten Farbpunkte des Beamers zu sehen. Deutlich zu erkennen ist, dass das Licht des Beamers auch für rein blaue bzw. rein rote Bilder einen hohen Anteil grünen Lichtes beinhaltet. Dies konnte bereits von den Teilnehmenden vor der Versuchsdurchführung beobachtet werden, da der Beamer einen deutlichen Grünstich besaß. Die Messwerte selbst sind jedoch mit geringen Standardabweichungen durchaus plausibel, es scheinen nur geringfügige Messfehler vor zu liegen. Erklärt werden kann dieser Grünstich durch einer Alterung der Beamer-Elemente. Vor allem die LCD-Elemente werden durch Hitze beeinträchtigt, und verlieren ihre Funktionsweise. So ist hier anscheinend der LCD des grünen Lichtstrahls zu durchlässig geworden. 
Der Flächeninhalt des Beamers ist 19\% des Adobe-Farbraumes, bzw. 27\% des sRGB-Farbraumes