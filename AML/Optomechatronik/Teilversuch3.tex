
In den ersten zwei Teilversuchen wurden die für die Farbgebung eines Beamers wichtigen Bauteile untersucht. Der dritte Teilversuch befasst sich nun mit der Analyse eines vollständigen Beamers und dessen Farbwiedergabe, um reale Werte für den Vergleich mit den berechneten Erwartungswerten zu liefern.

Mithilfe eines Goniometers und einer Leuchtdichtekamera werden die drei Farbkomponenten des Beamers über dessen Projektionsfläche gemittelt berechnet, und das darstellbare Farbdreieck wird berechnet.

\subsubsection{Versuchsaufbau}

Der Aufbau besteht aus einem Goniometer, d.h. einer automatisch drehbaren Platform, auf welche der Beamer befestigt ist. Der Beamer ist auf eine Leuchtdichtekamera gerichtet, welche mithilfe verschiedener Filter die Rot-, Grün- und Blauanteile des vom Beamer ausgestrahlten Lichtes misst. Ein Computer wird verwendet um die Spektren der Leuchtdichtekamera zu speichern, und um den Beamer und das Goniometer an zu steuern.

\subsubsection{Versuchsdurchführung}

