
In den ersten zwei Teilversuchen wurden die für die Farbgebung eines Beamers wichtigen Bauteile untersucht. Der dritte Teilversuch befasst sich nun mit der Analyse eines vollständigen Beamers und dessen Farbwiedergabe, um reale Werte für den Vergleich mit den berechneten Erwartungswerten zu liefern.

Mithilfe eines Goniometers und einer Leuchtdichtekamera werden die drei Farbkomponenten des Beamers über dessen Projektionsfläche gemittelt berechnet, um den Teilnehmenden eine Berechnung des realen Farbraumes zu ermöglichen.

\subsubsection{Versuchsaufbau}

Der Aufbau besteht aus einem Goniometer, d.h. einer automatisch drehbaren Platform, auf welche der Beamer befestigt ist. Der Beamer ist auf eine Leuchtdichtekamera gerichtet, welche mithilfe verschiedener Filter die Rot-, Grün- und Blauanteile des vom Beamer ausgestrahlten Lichtes misst. Ein Computer wird verwendet um die Spektren der Leuchtdichtekamera zu speichern, und um den Beamer und das Goniometer an zu steuern.

\subsubsection{Versuchsdurchführung}

Die Messung der drei Farbkomponenten wird in den folgenden Schritten durchgeführt:
\begin{itemize}
\item Die Belichtungszeit der Leuchtdichtekamera wird für die drei Farben so bestimmt, dass nicht im Sättigungsbereich des Sensors gemessen wird. Für die Teilnehmenden wurden bereits bestimmte Werte für die drei Farben vorgegeben.
\item Es werden 15 Messwinkel, welche möglichst gleichmäßig über die Projektionsfläche des Beamers verteilt sind, ausgewählt. Die Teilnehmenden wählten Winkel von $+15^\circ;\pm2.5^\circ$ aus, wobei insgesamt 15 Winkel vermessen wurden.
\item Es wird ein PDF einer der drei Farben (Rot, Grün oder Blau) über den Beamer projiziert, und für die gewählte Farbe über die 15 Messwinkel gemittelt. Der berechnete Farbpunkt wird abgespeichert, und dieser Schritt wird für alle drei Farben ausgeführt.
\end{itemize}

Zu beachten ist, dass das PDF mit den Farbbildern nicht unbedingt 100\% die Farbkanäle des Beamers ansteuert. Es gibt so z.B. immer eine gewisse Farbraumumrechnung, welche die Messwerte der Farbpositionen beeinträchtigen kann.